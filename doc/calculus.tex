\documentclass{article}[11pt]

\usepackage{amsmath}
\usepackage{amsfonts}
\usepackage{proof}
\usepackage{stmaryrd}

\begin{document}
\section{A Confession}
I suck at \LaTeX.

\section{Syntax}
$$
\begin{aligned}
x,y,z,w \in \mathit{Variable}\\
t,t',t'' \in \mathit{Type} &::= ()~ |~ t + t'~ |~ t \times t'~ |~ t \rightarrow t'~ |~ M(t)\\
d \in \mathit{DataType} &::= \text{a restriction of}~ \mathit{Type}~ \text{to ``data-only'' types}\\
M \in \mathit{Monad} &::= \mathbb{I}~ |~ T M\\
T \in \mathit{MonadTrans} &::= \mathbb{R}_{t\rightarrow{}t'}~ |~ \mathbb{S}_t~ |~ \mathbb{E}_t~ |~ \mathbb{M}_{d\rightarrow{}t}\\
e,e',e'' \in \mathit{Exp}&::= x~ |~ \lambda{}x_t\rightarrow{}e~ |~ e~ e'&&\text{(lambda calculus)}\\
  &~~~~|~ \textbf{nil}&&\text{(unit)}\\
  &~~~~|~ \textbf{inl}_{t,t'}~ |~ \textbf{inr}_{t,t'}~ |~ \textbf{case}~ e~ \textbf{of}~ \textbf{inl}~ x \rightarrow e'~ ; \textbf{inr}~ y \rightarrow e''~ \textbf{end}&&\text{(coproduct)}\\
  &~~~~|~ \textbf{mkpair}_{t,t'}~ |~ \pi^1_{t,t'}~ |~ \pi^2_{t,t'}&&\text{(product)}\\
  &~~~~|~ \textbf{return}_{t,M}~ |~ \textbf{bind}_{t,t',M}~ |~ \textbf{lift}_{t,M,T}&&\text{(monad operators)}\\
  &~~~~|~ \textbf{signal}_{t,t',M}&&\text{(reactive prims.)}\\
  &~~~~|~ \textbf{get}_{t,M}~ |~ \textbf{put}_{t,M}&&\text{(state prims.)}\\
  &~~~~|~ \textbf{throw}_{t,t',M}~ |~ \textbf{catch}_{t,t',M}&&\text{(error prims.)}\\
  &~~~~|~ \textbf{rdmem}_{t,t',M}~ |~ \textbf{wrmem}_{t,t',M}&&\text{(memory prims.)}
\end{aligned}
$$

\subsection{Notes}
Note: $\mathbb{R}$ is {\it ReactT}, $\mathbb{S}$ is {\it StateT}, $\mathbb{E}$ is {\it ErrorT}, and $\mathbb{M}$ is {\it MemT} (for addressable memory, which may or may not actually be synthesized as RAM).

In reality we're going to support non-recursive, first-order data types with named constructors, but for the abstract presentation, sums/products/unit will do (e.g. $Bit$ can be encoded as $() + ()$).

For the moment I am trying to avoid polymorphism entirely, meaning any expression constructor that on its own might have polymorphic type has a type subscript. The ``real'' language could have Hindley-Milner style polymorphism and of course on paper it's fine to leave off subscripts when they can be inferred from context.

The restriction of the domain type in $\mathbb{M}$ is not accidental. It's there because even in a higher-order setting we need to be able to test for equality.

It may not be necessary to make {\it MemT} a special primitive; memory extraction could probably be performed from {\it StateT}.

\section{Type System}
\infer[\textsc{T-Lam}]{\Gamma\vdash{}\lambda{}x_t\rightarrow{}e : t \rightarrow t'}{\Gamma,x : t\vdash{}e : t'}
\vspace{1em}
\infer[\textsc{T-Var}]{\Gamma,x : t\vdash{}x : t}{}
\vspace{1em}
\infer[\textsc{T-App}]{\Gamma\vdash{}e~ e' : t}{\Gamma\vdash{}e : t' \rightarrow t~~~\Gamma\vdash{}e' : t'}
\vspace{1em}
\infer[\textsc{T-Nil}]{\Gamma\vdash{}\textbf{nil} : ()}{}
\vspace{1em}
\infer[\textsc{T-InL}]{\Gamma\vdash{}\textbf{inl}_{t,t'} : t \rightarrow t + t'}{}
\vspace{1em}
\infer[\textsc{T-InR}]{\Gamma\vdash{}\textbf{inr}_{t,t'} : t' \rightarrow t + t'}{}
\vspace{1em}
\infer[\textsc{T-Case}]{\Gamma\vdash{}\textbf{case}~ e~ \textbf{of}~ \textbf{inl}~ x \rightarrow e'~ ; \textbf{inr}~ y \rightarrow e''~ \textbf{end} : t}{\Gamma\vdash{}e : t' + t''~~~~\Gamma,x : t'\vdash{}e' : t~~~~\Gamma,y : t''\vdash{}e'' : t}
\vspace{1em}
\infer[\textsc{T-MkPair}]{\Gamma\vdash{}\textbf{mkpair}_{t,t'} : t \rightarrow t' \rightarrow t \times t'}{}
\vspace{1em}
\infer[\textsc{T-Pi1}]{\Gamma\vdash{}\pi^1_{t,t'} : t \times t' \rightarrow t}{}
\vspace{1em}
\infer[\textsc{T-Pi2}]{\Gamma\vdash{}\pi^2_{t,t'} : t \times t' \rightarrow t'}{}
\vspace{1em}
\infer[\textsc{T-Return}]{\Gamma\vdash{}\textbf{return}_{t,M} : t \rightarrow M(t)}{}
\vspace{1em}
\infer[\textsc{T-Bind}]{\Gamma\vdash{}\textbf{bind}_{t,t',M} : M(t) \rightarrow (t \rightarrow M(t')) \rightarrow M(t')}{}
\vspace{1em}
\infer[\textsc{T-Lift}]{\Gamma\vdash{}\textbf{lift}_{t,M,T} : M(t) \rightarrow TM(t)}{}
\vspace{1em}
\infer[\textsc{T-Signal}]{\Gamma\vdash{}\textbf{signal}_{t,t',M} : t' \rightarrow \mathbb{R}_{t\rightarrow{}t'}M(t)}{}
\vspace{1em}
\infer[\textsc{T-Get}]{\Gamma\vdash{}\textbf{get}_{t,M} : \mathbb{S}_tM(t)}{}
\vspace{1em}
\infer[\textsc{T-Put}]{\Gamma\vdash{}\textbf{put}_{t,M} : t \rightarrow \mathbb{S}_tM\big(()\big)}{}
\vspace{1em}
\infer[\textsc{T-Throw}]{\Gamma\vdash{}\textbf{throw}_{t,t',M} : t' \rightarrow \mathbb{E}_{t'}M(t)}{}
\vspace{1em}
\infer[\textsc{T-Catch}]{\Gamma\vdash{}\textbf{catch}_{t,t',M} : \mathbb{E}_{t'}M(t) \rightarrow (t' \rightarrow \mathbb{E}_{t'}M(t)) \rightarrow \mathbb{E}_{t'}M(t)}{}
\vspace{1em}
\infer[\textsc{T-RdMem}]{\Gamma\vdash{}\textbf{rdmem}_{t,t',M} : t \rightarrow \mathbb{M}_{t\rightarrow{}t'}M(t')}{}
\vspace{1em}
\infer[\textsc{T-WrMem}]{\Gamma\vdash{}\textbf{wrmem}_{t,t',M} : t \rightarrow t' \rightarrow \mathbb{M}_{t\rightarrow{}t'}M\big(()\big)}{}

\section{Denotational Semantics}

\subsection{Of Types}
$\llbracket{}()\rrbracket = \mathbf{1}$

$\llbracket{}t+t'\rrbracket = \llbracket{}t\rrbracket + \llbracket{}t'\rrbracket$

$\llbracket{}t\times{}t'\rrbracket = \llbracket{}t\rrbracket \times \llbracket{}t'\rrbracket$

$\llbracket{}t\rightarrow{}t'\rrbracket = \llbracket{}t'\rrbracket^{\llbracket{}t\rrbracket}$

$\llbracket{}M(t)\rrbracket = \llbracket{}M\rrbracket\llbracket{}t\rrbracket$

\subsection{Of Monads}
$\llbracket{}\mathbb{I}\rrbracket = \lambda{}t . t$

$\llbracket{}T M\rrbracket = \llbracket{}T\rrbracket \llbracket{}M\rrbracket$

\subsection{Of Monad Transformers}
$\llbracket{}\mathbb{R}_{t\rightarrow{}t'}\rrbracket = \lambda{} M . \lambda{} t'' . \nu{}X . M (t'' + (\llbracket{}t'\rrbracket \times (\llbracket{}t\rrbracket \rightarrow X)))$

$\llbracket{}\mathbb{S}_t\rrbracket = \lambda{} M . \lambda{} t' . \llbracket{}t\rrbracket \rightarrow M (t' \times \llbracket{}t\rrbracket)$

$\llbracket{}\mathbb{E}_t\rrbracket = \lambda{} M . \lambda{} t' . M (\llbracket{}t\rrbracket + t')$

$\llbracket{}\mathbb{M}_{t\rightarrow{}t'}\rrbracket = \lambda{} M . \lambda{} t'' . (\llbracket{}t\rrbracket{}\rightarrow\llbracket{}t'\rrbracket) \rightarrow M (t'' \times (\llbracket{}t\rrbracket{}\rightarrow\llbracket{}t'\rrbracket))$

\section{Translation Scheme to RWNF}

\end{document}
