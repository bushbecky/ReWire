\documentclass{article}[11pt]

\usepackage{amsmath}
\usepackage{amsfonts}
\usepackage{proof}
\usepackage{stmaryrd}

\begin{document}
\section{Syntax}

\subsection{Signals}
$$
\begin{aligned}
x,y,z,w,x_i \in \mathit{Variable}\\
t,t',t'',t''',t_i \in \mathit{SigType} &::= ()~ |~ t + t'~ |~ t \times t'\\
e,e',e'',e_i \in \mathit{SigExp}&::= x~ e_1~ \dots~ e_n&&n \ge 0\\
  &~~~~|~ \mathbf{let}~ x = e~ \mathbf{in}~ e'~ \mathbf{end}\\
  &~~~~|~ \mathbf{nil}~ |~ \mathbf{inl}_t~ e~ |~ \mathbf{inr}_t~ e~ |~ \mathbf{mkpair}~ e~ e'\\
  &~~~~|~ \mathbf{case}~ e~ \mathbf{of}~ \mathbf{inl}~ x \rightarrow e'~ ; \mathbf{inr}~ y \rightarrow e''~ \mathbf{end}\\
  &~~~~|~ \mathbf{fst}~ e~ |~ \mathbf{snd}~ e
\end{aligned}
$$

\subsection{Computations}
$$
\begin{aligned}
T \in \mathit{MonadTrans} &::= \mathbb{S}_t~ |~ \mathbb{M}_{t:t'}\\
B \in \mathit{BaseMonad} &::= \mathbb{I}~ |~ T B\\
e_b,e'_b \in \mathit{BaseExp} &::= x~ e_1~ \dots~ e_n&&n \ge 0\\
  &~~~~|~ \mathbf{let}~ x = e~ \mathbf{in}~ e_b~ \mathbf{end}\\
  &~~~~|~ \mathbf{case}~ e~ \mathbf{of}~ \mathbf{inl}~ x \rightarrow e_b~ ; \mathbf{inr}~ y \rightarrow e'_b~ \mathbf{end}\\
  &~~~~|~ \mathbf{return}_B~ e~ |~ \mathbf{bind}~ x \leftarrow{} e_b~ \mathbf{in}~ e'_b~ \mathbf{end}\\
  &~~~~|~ \mathbf{lift}_T~ e_b\\
  &~~~~|~ \mathbf{get}_B~ |~ \mathbf{put}_B~ e\\
  &~~~~|~ \mathbf{rdmem}_B~ e~ |~ \mathbf{wrmem}_B~ e~ e'\\
R \in \mathit{ReMonad} &::= \mathbb{R}_{t:t'} B~ |~ T R\\
e_r,e'_r \in \mathit{ReExp} &::= x~ e_1~ \dots~ e_n&&n \ge 0\\
  &~~~~|~ \mathbf{let}~ x = e~ \mathbf{in}~ e_r~ \mathbf{end}\\
  &~~~~|~ \mathbf{case}~ e~ \mathbf{of}~ \mathbf{inl}~ x \rightarrow e_r~ ; \mathbf{inr}~ y \rightarrow e'_r~ \mathbf{end}\\
  &~~~~|~ \mathbf{return}_R~ e~ |~ \mathbf{bind}~ x \leftarrow{} e_r~ \mathbf{in}~ e'_r~ \mathbf{end}\\
  &~~~~|~ \mathbf{lift}_{\mathbb{R}_{t:t'}}~ e_b\\
  &~~~~|~ \mathbf{lift}_T~ e_r\\
  &~~~~|~ \mathbf{signal}_R~ e\\
  &~~~~|~ \mathbf{get}_R~ |~ \mathbf{put}_R~ e\\
  &~~~~|~ \mathbf{rdmem}_R~ e~ |~ \mathbf{wrmem}_R~ e~ e'\\
\end{aligned}
$$

\subsection{Programs}
$$
\begin{aligned}
l \in \mathit{LetDefn}&::= x~ (x_1 : t_1)~ \dots~ (x_n : t_n) = e&&n \ge 0\\
  &~~~~|~ x~ (x_1 : t_1)~ \dots~ (x_n : t_n) = e_b&&n \ge 0\\
b,b' \in \mathit{MachineBody}&::= \mathbf{let}~ x = e~ \mathbf{in}~ b~ \mathbf{end}\\
  &~~~~|~ \mathbf{case}~ e~ \mathbf{of}~ \mathbf{inl}~ x \rightarrow b~ ; \mathbf{inr}~ y \rightarrow b'~ \mathbf{end}\\
  &~~~~|~ \mathbf{return}_R~ e~ |~ \mathbf{bind}~ x \leftarrow{} b~ \mathbf{in}~ b~ \mathbf{end}\\
  &~~~~|~ \mathbf{bind}~ x \leftarrow{} \mathbf{signal}_R~ e~ \mathbf{in}~ e_r~ \mathbf{end}\\
  &~~~~|~ \mathbf{lift}_{\mathbb{R}_{t:t'}}~ e_b\\
  &~~~~|~ \mathbf{lift}_T~ b\\
  &~~~~|~ \mathbf{signal}_R~ e\\
m_i \in \mathit{MachineDefn}&::= x~ (x_1 : t_1)~ \dots~ (x_n : t_n) = b&&n \ge 0\\
P \in \mathit{Prog}&::= \mathbf{let}~ l~ \mathbf{in}~ P~ \mathbf{end}\\
  &~~~~|~ \mathbf{letrec}~ m_1~ \dots~ m_n~ \mathbf{in}~ x~ \mathbf{end}&&n \ge 1
\end{aligned}
$$

\subsection{Notes}
$\mathbb{R}$ is {\it ReactT}, $\mathbb{S}$ is {\it StateT}, and $\mathbb{M}$ is {\it MemT} (for addressable memory, which may or may not actually be synthesized as RAM).

In reality we're going to support non-recursive, first-order data types with named constructors, but for the abstract presentation, sums/products/unit will do (e.g. $Bit$ can be encoded as $() + ()$).

For the moment I am trying to avoid polymorphism entirely, meaning any expression constructor that on its own might have polymorphic type has a type subscript. The ``real'' language could have Hindley-Milner style polymorphism and of course on paper it's fine to leave off subscripts when they can be inferred from context.

It may not be necessary to make {\it MemT} a special primitive; memory extraction could probably be performed from {\it StateT}.

\section{Type System}
\infer[\textsc{T-Var}]{\Gamma,x : t\vdash{}x : t}{}
\vspace{1em}
\infer[\textsc{T-Lam}]{\Gamma\vdash{}\lambda{}x_t\rightarrow{}e : t \rightarrow t'}{\Gamma,x : t\vdash{}e : t'}
\vspace{1em}
\infer[\textsc{T-App}]{\Gamma\vdash{}e~ e' : t}{\Gamma\vdash{}e : t' \rightarrow t~~~\Gamma\vdash{}e' : t'}
\vspace{1em}
\infer[\textsc{T-Let}]{\Gamma\vdash{}\textbf{let}~ x_t = e~ \textbf{in}~ e'~ \textbf{end} : t'}{\Gamma,x : t\vdash{}e' : t'~~~\Gamma \vdash e : t}
\vspace{1em}
\infer[\textsc{T-Nil}]{\Gamma\vdash{}\textbf{nil} : ()}{}
\vspace{1em}
\infer[\textsc{T-InL}]{\Gamma\vdash{}\textbf{inl}_{t,t'} : t \rightarrow t + t'}{}
\vspace{1em}
\infer[\textsc{T-InR}]{\Gamma\vdash{}\textbf{inr}_{t,t'} : t' \rightarrow t + t'}{}
\vspace{1em}
\infer[\textsc{T-Case}]{\Gamma\vdash{}\textbf{case}~ e~ \textbf{of}~ \textbf{inl}~ x \rightarrow e'~ ; \textbf{inr}~ y \rightarrow e''~ \textbf{end} : t}{\Gamma\vdash{}e : t' + t''~~~~\Gamma,x : t'\vdash{}e' : t~~~~\Gamma,y : t''\vdash{}e'' : t}
\vspace{1em}
\infer[\textsc{T-MkPair}]{\Gamma\vdash{}\textbf{mkpair}_{t,t'} : t \rightarrow t' \rightarrow t \times t'}{}
\vspace{1em}
\infer[\textsc{T-Pi1}]{\Gamma\vdash{}\mathbf{fst}_{t,t'} : t \times t' \rightarrow t}{}
\vspace{1em}
\infer[\textsc{T-Pi2}]{\Gamma\vdash{}\mathbf{snd}_{t,t'} : t \times t' \rightarrow t'}{}
\vspace{1em}
\infer[\textsc{T-Return}]{\Gamma\vdash{}\textbf{return}_{t,M} : t \rightarrow M(t)}{}
\vspace{1em}
\infer[\textsc{T-Bind}]{\Gamma\vdash{}\textbf{bind}_{t,t',M} : M(t) \rightarrow (t \rightarrow M(t')) \rightarrow M(t')}{}
\vspace{1em}
\infer[\textsc{T-Lift}]{\Gamma\vdash{}\textbf{lift}_{t,M,T} : M(t) \rightarrow TM(t)}{}
\vspace{1em}
\infer[\textsc{T-Signal}]{\Gamma\vdash{}\textbf{signal}_{t,t',M} : t' \rightarrow \mathbb{R}_{t:t'}M(t)}{}
\vspace{1em}
\infer[\textsc{T-Unfold}]{\Gamma\vdash{}\textbf{unfold}_{t,t',t'',t''',M} : t \rightarrow (t \rightarrow M (t' + (t\times{}t''\times{}(t''' \rightarrow t)))) \rightarrow \mathbb{R}_{t''':t''}M t'}{}
\vspace{1em}
\infer[\textsc{T-Get}]{\Gamma\vdash{}\textbf{get}_{t,M} : \mathbb{S}_tM(t)}{}
\vspace{1em}
\infer[\textsc{T-Put}]{\Gamma\vdash{}\textbf{put}_{t,M} : t \rightarrow \mathbb{S}_tM\big(()\big)}{}
\vspace{1em}
\infer[\textsc{T-Throw}]{\Gamma\vdash{}\textbf{throw}_{t,t',M} : t' \rightarrow \mathbb{E}_{t'}M(t)}{}
\vspace{1em}
\infer[\textsc{T-Catch}]{\Gamma\vdash{}\textbf{catch}_{t,t',M} : \mathbb{E}_{t'}M(t) \rightarrow (t' \rightarrow \mathbb{E}_{t'}M(t)) \rightarrow \mathbb{E}_{t'}M(t)}{}
\vspace{1em}
\infer[\textsc{T-RdMem}]{\Gamma\vdash{}\textbf{rdmem}_{t,t',M} : t \rightarrow \mathbb{M}_{t\rightarrow{}t'}M(t')}{}
\vspace{1em}
\infer[\textsc{T-WrMem}]{\Gamma\vdash{}\textbf{wrmem}_{t,t',M} : t \rightarrow t' \rightarrow \mathbb{M}_{t\rightarrow{}t'}M\big(()\big)}{}

\section{Denotational Semantics}

\subsection{Of Types}
\begin{eqnarray*}
\llbracket{}()\rrbracket &=& \{()\}\\
\llbracket{}t+t'\rrbracket &=& \llbracket{}t\rrbracket + \llbracket{}t'\rrbracket\\
\llbracket{}t\times{}t'\rrbracket &=& \llbracket{}t\rrbracket \times \llbracket{}t'\rrbracket\\
\llbracket{}t\rightarrow{}t'\rrbracket &=& \llbracket{}t'\rrbracket^{\llbracket{}t\rrbracket}\\
\llbracket{}M(t)\rrbracket &=& \mathcal{M}\llbracket{}M\rrbracket\llbracket{}t\rrbracket
\end{eqnarray*}

\subsection{Of Monads}

\subsubsection{Type Constructors}
\begin{eqnarray*}
\mathcal{M}\llbracket{}\mathbb{I}\rrbracket &=& \lambda t . t\\
\mathcal{M}\llbracket{}\mathbb{R}_{t:t'}M\rrbracket &=& \lambda t'' . \nu{}X . \mathcal{M}\llbracket{}M\rrbracket (t'' + (\llbracket{}t'\rrbracket \times (\llbracket{}t\rrbracket \rightarrow X)))\\
\mathcal{M}\llbracket{}\mathbb{S}_t M\rrbracket &=& \lambda t' . \llbracket{}t\rrbracket \rightarrow \mathcal{M}\llbracket{}M\rrbracket (t' \times \llbracket{}t\rrbracket)\\
\mathcal{M}\llbracket{}\mathbb{E}_t M\rrbracket &=& \lambda{} t' . \mathcal{M}\llbracket{}M\rrbracket (\llbracket{}t\rrbracket + t')\\
\mathcal{M}\llbracket{}\mathbb{M}_{t:t'} M\rrbracket &=& \mathcal{M}\llbracket{}\mathbb{S}_{t\rightarrow{}t'} M\rrbracket\\
\end{eqnarray*}

\subsubsection{Unit Operators}
\begin{eqnarray*}
\mathcal{U}\llbracket{}\mathbb{I}\rrbracket &=& \lambda x . x\\
\mathcal{U}\llbracket{}\mathbb{R}_{t:t'}M\rrbracket &=& \mathcal{U}\llbracket{}M\rrbracket \circ \mathbf{inl}\\
\mathcal{U}\llbracket{}\mathbb{S}_t M\rrbracket &=& \lambda x . \lambda \sigma . \mathcal{U}\llbracket{}M\rrbracket \langle x,\sigma \rangle\\
\mathcal{U}\llbracket{}\mathbb{E}_t M\rrbracket &=& \mathcal{U}\llbracket{}M\rrbracket \circ \mathbf{inr}\\
\mathcal{U}\llbracket{}\mathbb{M}_{t:t'} M\rrbracket &=& \mathcal{U}\llbracket{}\mathbb{S}_{t\rightarrow{}t'} M\rrbracket\\
\end{eqnarray*}

\subsubsection{Bind Operators}
\begin{eqnarray*}
\mathcal{B}\llbracket{}\mathbb{I}\rrbracket &=& \lambda x . \lambda f . f x\\
\mathcal{B}\llbracket{}\mathbb{R}_{t:t'}M\rrbracket &=& \mathbf{fix}~ F . \lambda \varphi . \lambda f . \mathcal{B}\llbracket{}M\rrbracket \varphi \left(\lambda r . \begin{cases}f x&\text{if } r = \mathbf{inl}~ x\\\mathcal{U}\llbracket{}M\rrbracket (\mathbf{inr} \langle o,\lambda i . F~ (\kappa~ i)~ f\rangle)&\text{if } r = \mathbf{inr} \langle{}o,\kappa\rangle\end{cases}\right)\\
\mathcal{B}\llbracket{}\mathbb{S}_t M\rrbracket &=& \lambda \varphi . \lambda f . \lambda s . \mathcal{B}\llbracket{}M\rrbracket (\varphi s) (\lambda r . f (\pi_2r) (\pi_1 r))\\
\mathcal{B}\llbracket{}\mathbb{E}_t M\rrbracket &=& \lambda \varphi . \lambda f . \mathcal{B}\llbracket{}M\rrbracket \varphi \left(\lambda r . \begin{cases}\mathcal{U}\llbracket{}M\rrbracket{}r&\text{if } r = \mathbf{inl}~ e\\f x&\text{if } r = \mathbf{inr}~ x\end{cases}\right)\\
\mathcal{B}\llbracket{}\mathbb{M}_{t:t'} M\rrbracket &=& \mathcal{B}\llbracket{}\mathbb{S}_{t\rightarrow{}t'} M\rrbracket\\
\end{eqnarray*}

\subsubsection{Lift Operators}
\begin{eqnarray*}
\mathcal{L}\llbracket{}M,\mathbb{R}_{t:t'}\rrbracket &=& \lambda \varphi . \mathcal{B}\llbracket{}M\rrbracket \varphi (\mathcal{U}\llbracket{}M\rrbracket \circ \mathbf{inl})\\
\mathcal{L}\llbracket{}M,\mathbb{S}_t\rrbracket &=& \lambda \varphi . \lambda \sigma . \mathcal{B}\llbracket{}M\rrbracket \varphi (\lambda x . \mathcal{U}\llbracket{}M\rrbracket\langle{}x,\sigma\rangle)\\
\mathcal{L}\llbracket{}M,\mathbb{E}_t\rrbracket &=& \lambda \varphi . \mathcal{B}\llbracket{}M\rrbracket \varphi (\mathcal{U}\llbracket{}M\rrbracket \circ \mathbf{inr})\\
\mathcal{L}\llbracket{}M,\mathbb{M}_{t:t'}\rrbracket &=& \mathcal{L}\llbracket{}M,\mathbb{S}_{t\rightarrow{}t'}\rrbracket
\end{eqnarray*}

\subsection{Of Expressions}
For clarity's sake (?), irrelevant type subscripts are dropped.

\begin{eqnarray*}
\llbracket{}x\rrbracket{}\rho &=& \rho x\\
\llbracket{}\lambda x_t \rightarrow e\rrbracket \rho &=& \lambda v \rightarrow \llbracket{}e\rrbracket(\rho[x\mapsto{}v])\\
\llbracket{}e~ e'\rrbracket \rho &=& \llbracket{}e\rrbracket\rho~ (\llbracket{}e'\rrbracket\rho)\\
\llbracket{}\mathbf{let}~ x = e~ \mathbf{in}~ e' \mathbf{end}\rrbracket \rho &=& \llbracket{}e'\rrbracket (\rho[x\mapsto{}\llbracket{}e\rrbracket\rho])\\
\llbracket{}\mathbf{nil}\rrbracket \rho &=& ()\\
\llbracket{}\mathbf{inl}\rrbracket \rho &=& \mathbf{inl}\\
\llbracket{}\mathbf{inr}\rrbracket \rho &=& \mathbf{inr}\\
\llbracket{}\mathbf{case}~ e~ \mathbf{of}~ \mathbf{inl}~ x \rightarrow e' ~; \mathbf{inr}~ y \rightarrow e''~ \mathbf{end}\rrbracket \rho &=& \begin{cases}
                                                                                                                                 \llbracket e' \rrbracket (\rho[x\mapsto{}v])&\text{if } \llbracket e \rrbracket \rho = \mathbf{inl}~ v\\
                                                                                                                                 \llbracket e'' \rrbracket (\rho[y\mapsto{}v])&\text{if } \llbracket e \rrbracket \rho = \mathbf{inr}~ v\\
                                                                                                                                \end{cases}\\
\llbracket{}\mathbf{mkpair}\rrbracket{}\rho &=& \lambda x . \lambda y . \langle x,y \rangle\\
\llbracket{}\mathbf{fst}\rrbracket\rho &=& \pi_1\\
\llbracket{}\mathbf{snd}\rrbracket\rho &=& \pi_2\\
\llbracket{}\mathbf{return}_M\rrbracket{}\rho &=& \mathcal{U}\llbracket{}M\rrbracket\\
\llbracket{}\mathbf{bind}_M\rrbracket{}\rho &=& \mathcal{B}\llbracket{}M\rrbracket\\
\llbracket{}\mathbf{lift}_{M,T}\rrbracket\rho &=& \mathcal{L}\llbracket{}M,T\rrbracket\\
\llbracket{}\mathbf{signal}_M\rrbracket\rho &=& \lambda o . \mathcal{U}\llbracket{}M\rrbracket(\mathbf{inr}~ \langle o,\mathcal{U}\llbracket{}\mathbb{R}M\rrbracket\rangle)\\
\llbracket{}\mathbf{unfold}_M\rrbracket\rho &=& \mathbf{fix}~ F . \lambda s . \lambda f . \mathbb{B}\llbracket{}M\rrbracket~(f s)~ \left(\lambda r . \begin{cases}\mathbb{U}\llbracket{}M\rrbracket~ (\mathbf{inr} \langle{}o,\lambda i . F~ s'~ f\rangle{})\\~~~\text{when } r = \mathbf{inr}\langle{}s',o,k\rangle\\\mathbb{U}\llbracket{}M\rrbracket~ (\mathbf{inr}~ a)\\~~~\text{when } r = \mathbf{inl}~ a\end{cases}\right)\\
\llbracket{}\mathbf{get}_M\rrbracket\rho &=& \lambda \sigma . \mathcal{U}\llbracket{}M\rrbracket\langle\sigma,\sigma\rangle\\
\llbracket{}\mathbf{put}_M\rrbracket\rho &=& \lambda \sigma' . \lambda \sigma . \mathcal{U}\llbracket{}M\rrbracket\langle(),\sigma'\rangle\\
\llbracket{}\mathbf{throw}_M\rrbracket\rho &=& \mathcal{U}\llbracket{}M\rrbracket \circ \mathbf{inl}\\
\llbracket{}\mathbf{catch}_M\rrbracket\rho &=& \lambda h . \lambda \varphi . \mathcal{B}\llbracket{}M\rrbracket\varphi\left(\lambda r . \begin{cases}
                                                                                                                                               h~ e&\text{if }r = \mathbf{inl}~ e\\
                                                                                                                                               \mathcal{U}\llbracket{}M\rrbracket r&\text{otherwise}
                                                                                                                                              \end{cases}\right)\\
\llbracket{}\mathbf{rdmem}_M\rrbracket\rho &=& \lambda \alpha . \lambda \mu . \mathcal{U}\llbracket{}M\rrbracket \langle\mu\alpha,\mu\rangle\\
\llbracket{}\mathbf{wrmem}_M\rrbracket\rho &=& \lambda \alpha . \lambda \omega . \lambda \mu . \mathcal{U}\llbracket{}M\rrbracket \langle(),\mu[\alpha \mapsto \omega]\rangle
\end{eqnarray*}

\section{Monadic Normal Form}
A monadic normal form program is a typed system of equations:

\begin{eqnarray*}
\varphi_0 &::& D \rightarrow \mathbb{R}_{D:D'} M D''\\
\varphi_0 &=& \lambda i \rightarrow e_0\\
\varphi_1 &::& D_{1,0} \rightarrow \cdots \rightarrow D_{1,m_1} \rightarrow D \rightarrow \mathbb{R}_{D:D'} M D''\\
\varphi_1 &=& \lambda x_0 \rightarrow \cdots \lambda x_{m_1} \rightarrow \lambda i \rightarrow e_1\\
          &\vdots&\\
\varphi_n &::& D_{n,0} \rightarrow \cdots \rightarrow D_{n,m_n} \rightarrow D \rightarrow \mathbb{R}_{D:D'} M D''\\
\varphi_n &=& \lambda x_0 \rightarrow \cdots \lambda x_{m_n} \rightarrow \lambda i \rightarrow e_n
\end{eqnarray*}
%
where $D$, $D'$, $D''$, and each $D_{i,j}$ are representable types, and each $e_i$ conforms to $\mathit{ebody}$ in the grammar:
%
$$
\begin{aligned}
\mathit{ebody} &::= \mathbf{case}~ \mathit{earg}~ \mathbf{of}~ \mathbf{inl}~ x \rightarrow \mathit{ebody} ; \mathbf{inr}~ y \rightarrow \mathit{ebody}~ \mathbf{end}\\
               &~~~~|~ \mathit{enext}\\
\mathit{earg}  &::= \mathit{x}~ |~ \mathbf{nil}~ |~ \mathbf{inl}~ \mathit{earg}~ |~ \mathbf{inr}~ \mathit{earg}~ |~ \mathbf{case}~ \mathit{earg}~ \mathbf{of}~ \mathbf{inl}~ x \rightarrow \mathit{earg} ; \mathbf{inr}~ y \rightarrow \mathit{earg}~ \mathbf{end}\\
               &~~~~|~ \mathbf{mkpair}~ \mathit{earg}~ \mathit{earg}~ |~ \mathbf{fst}~ \mathit{earg}~ |~ \mathbf{snd}~ \mathit{earg}\\
\mathit{enext} &::= \mathbf{bind}~ (\mathbf{signal}~ \mathit{earg})~ (\varphi_i~ \mathit{earg}_0~ \cdots \mathit{earg}_{m_i})\\
               &~~~~|~ \mathbf{return}~ \mathit{earg}
\end{aligned}
$$

Notably:
\begin{itemize}
\item Anything in this form is {\em productive} starting from $\varphi_0~ i$.
\item This form disallows the construction or destruction of recursive structures at runtime, {\em except} for the resumption representing the state machine.
\item This form disallows lambdas in general, permitting them only when used to introduce the parameters of $\varphi_i$.
\item Again, in the ``wild'' we could allow non-recursive data types and ``simple'' lambdas (non-recursive functions on data).
\item I have allowed ``done'' ($\mathbf{return}$) because hey, why not?
\item It so happens (I think!) that one cannot construct a well-typed thing with $\mathbb{R}$ lurking inside $M$; i.e. one cannot layer resumptions (which is good).
\item It is an extremely short hop to defunctionalize this to an application of $\mathbf{unfold}$.
\end{itemize}

\section{Translation to Monadic Normal Form}


\end{document}
